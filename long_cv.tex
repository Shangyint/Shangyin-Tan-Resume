%%%%%%%%%%%%%%%%%%%%%%%%%%%%%%%%%%%%%%%%%
% Medium Length Professional CV
% LaTeX Template
% Version 2.0 (8/5/13)
%
% This template has been downloaded from:
% http://www.LaTeXTemplates.com
%
% Original author:
% Trey Hunner (http://www.treyhunner.com/)
%
% Important note:
% This template requires the resume.cls file to be in the same directory as the
% .tex file. The resume.cls file provides the resume style used for structuring the
% document.
%
%%%%%%%%%%%%%%%%%%%%%%%%%%%%%%%%%%%%%%%%%

%----------------------------------------------------------------------------------------
%	PACKAGES AND OTHER DOCUMENT CONFIGURATIONS
%----------------------------------------------------------------------------------------

\documentclass{resume} % Use the custom resume.cls style
\usepackage{apacite}


\usepackage{bibentry}
\usepackage{hanging}
\newcommand\publication[1]{%
    \smallskip\par\hangpara{1.5em}{1}\bibentry{#1}\smallskip
}
\usepackage[left=0.75in,top=0.6in,right=0.75in,bottom=0.6in]{geometry} % Document margins
\usepackage{hyperref}
\name{Shangyin Tan} % Your name
\address{(+1) 765 - 427 - 2861\ \bf \\ tan279@purdue.edu \\ \url{https://github.com/Shangyint}} % Your phone number and email



\begin{document}
\nobibliography{refs}
\bibliographystyle{unsrt} 

%----------------------------------------------------------------------------------------
%	EDUCATION SECTION
%----------------------------------------------------------------------------------------

\begin{rSection}{Education}

\begin{rSubsection}{Purdue University}{2018 - 2022}{\textit{Bachelor of Science in Computer Science Honors}}{West Lafayette, US}
\item GPA: 3.97/4.0, \ Major GPA: 4.0
\item Corporate Partner Scholarship
\item PurPL Undergraduate Researcher
\item Graduate Courses: \it Algorithms, Programming Languages,
Program Reasoning, Numerical Analysis
\end{rSubsection}

\end{rSection}

%----------------------------------------------------------------------------------------
%	WORK EXPERIENCE SECTION
%----------------------------------------------------------------------------------------

\begin{rSection}{Recent Projects}

\begin{rSubsection}{Compiling Symbolic Execution}{May 2020 - Present}{\textit{Undergraduate Researcher} (advised by \href{http://continuation.passing.style/}{\texttt{Guannan Wei}} and \href{http://tiarkrompf.github.io/}{\texttt{Tiark Rompf}} )}{West Lafayette, US}
\item \url{https://github.com/Kraks/sai}
\item Build backend to generate SMT solver calls via metaprogramming
\item Develop \textbf{\it LLVM} symbolic compilation with free monads from scratch
\item Our paper \textit{Compiling Symbolic Execution with Staging and Algebraic Effects} is accepted at \\ \textbf{OOPSLA 2020}\ ! 
\end{rSubsection}

%------------------------------------------------

\begin{rSubsection}{$\mathbf{W}^2$: Synthesising Webpage from Wireframe}{March 2020 - Aug 2020}{\textit{Undergraduate Researcher} (advised by \href{https://www.cs.purdue.edu/homes/roopsha/}{\texttt{Roopsha Samanta}})}{West Lafayette, US}
\item \url{https://github.com/TigerHix/W2}
\item Design an algorithm to infer hierarchical layout from static structure
\item Transform static graph to responsive webpage (HTML)

\end{rSubsection}

\begin{rSubsection}{MiniScala: a Small Scala Compiler}{Jan 2020 - May 2020}{\textit{Developer}}{West Lafayette, US}
\item Parse and compile \textbf{Scala} source code to X86-64 assembly
\item Infer and check types of the input program
\item Optimize via Dead Code Elimination, Constant Folding, CPS Transformation, etc

\end{rSubsection}

\begin{rSubsection}{Gomuku Go}{July 2017}{\textit{Developer}}{Pittsburgh, US}
\item Design and build a Gomuku game 
\item Use Minimax algorithm to implement an AI player
\end{rSubsection}

%------------------------------------------------

\end{rSection}


\begin{rSection}{Publication}
\begin{enumerate}
    \item \bibentry{10.1145/3428232}
\end{enumerate}








\end{rSection}

\begin{rSection}{Presentation}


\textbf{PurPL Reading Group} \\
\textit{Data types a la carte}\hfill Aug 2020\\
\textit{A Lightweight Symbolic Virtual Machine for Solver-Aided Host Languages} \hfill Dec 2020



\end{rSection}


\begin{rSection}{Experience}

\begin{rSubsection}{Student Volunteer}{Nov 2020}{}{}
\item SPLASH 2020: Review talk videos. Monitor Q\&A sessions.


\end{rSubsection}
\begin{rSubsection}{Undergraduate Teaching Assistant}{Jan 2019 - Present}{\textit{Discrete Math, System Programming, Algorithms Analysis, ...}}{West Lafayette, US}
\item Conduct recitations to help students with problem solving
\item Advise students in lab debugging
\end{rSubsection}

\begin{rSubsection}{Selected Coding Contests}{2018 - 2020}{\it Higher Ranked Participant}{Midwest, US}
\item $3^{rd}$ in Tech Challenge Google 2019, Chicago
\item $2^{nd}$ in Sandia Coding Challenge 2018, West Lafayette

\end{rSubsection}


\end{rSection}

%----------------------------------------------------------------------------------------
%	TECHNICAL STRENGTHS SECTION



\begin{rSection}{Skills}

\begin{tabular}{ @{} >{\bfseries}l @{\hspace{6ex}} l }
Familiar with & C, Scala, Python, C++ \\
Have worked with & Haskell, Coq, X86-64, Java, Javascript, Scheme, \LaTeX, LLVM, MatLab\\
Tools & GDB, Linux, Bash, Git, SAT/SMT solvers (Minisat, STP, Z3)


\end{tabular}\\
(Skills in the same row are in random order)
\end{rSection}


%----------------------------------------------------------------------------------------
%	EXAMPLE SECTION
%----------------------------------------------------------------------------------------

%	Volunteering Activities and Projects
%----------------------------------------------------------------------------------------

%	TECHNICAL STRENGTHS SECTION
%----------------------------------------------------------------------------------------

%----------------------------------------------------------------------------------------

\end{document}
